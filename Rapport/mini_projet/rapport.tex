%%%%%%%%%%%%%%%%%%%%%%%%%%%%%%%%%%%%%%%%%
% University/School Laboratory Report
% LaTeX Template
% Version 4.0 (March 21, 2022)
%
% This template originates from:
% https://www.LaTeXTemplates.com
%
% Authors:
% Vel (vel@latextemplates.com)
% Linux and Unix Users Group at Virginia Tech Wiki
%
% License:
% CC BY-NC-SA 4.0 (https://creativecommons.org/licenses/by-nc-sa/4.0/)
%
%%%%%%%%%%%%%%%%%%%%%%%%%%%%%%%%%%%%%%%%%

%----------------------------------------------------------------------------------------
%	PACKAGES AND DOCUMENT CONFIGURATIONS
%----------------------------------------------------------------------------------------

\documentclass[
	a4paper, % Paper size, specify a4paper (A4) or letterpaper (US letter)
	10pt, % Default font size, specify 10pt, 11pt or 12pt
]{CSUniSchoolLabReport}


\addbibresource{sample.bib} % Bibliography file (located in the same folder as the template)

%----------------------------------------------------------------------------------------
%	REPORT INFORMATION
%----------------------------------------------------------------------------------------

\title{MA-CSEL \\ Conception système Embarqué Linux \\ Mini-Projet } % Report title

\author{Kirill \textsc{Goundiaev} \& Tanguy \textsc{Dietrich}} % Author name(s), add additional authors like: '\& James \textsc{Smith}'

\fancyhead[RE,LO]{Kirill Goundiaev \& Tanguy Dietrich}
\fancyhead[CE,CO]{\today}
\fancyhead[LE,RO]{MA-CSEL}

\fancyfoot[RE,LO]{MSE}
\fancyfoot[CE,CO]{\thepage} % this dicard page number dont touch
\fancyfoot[LE,RO]{HES-SO}
\date{\today} % Date of the report


% add image on the left top corner
% Definition of \maketitle
\makeatletter         
\def\@maketitle{
\raggedright
% \includegraphics[width = 60mm]{Figures/MSE.png}\\[8ex]
\includegraphics[width = 180mm]{Figures/ImageTitle.png}\\[8ex]
\begin{center}{}
{\Huge \@title }\\[4ex] 
{\Large  \@author}\\[4ex] 
\@date\\[8ex]
% \includegraphics[width = 40mm]{Figures/HESSO.png}
\end{center}}
\makeatother

%----------------------------------------------------------------------------------------

\begin{document}

\maketitle % Insert the title, author and date using the information specified above

% add an image
\begin{figure}[H] % Example image
\center{\includegraphics[width=0.35\linewidth]{EMbeddedLinuxLogo}}
% \caption{Example image.}
\label{fig:speciation}
\end{figure}

\begin{figure}[H] % Example image
	\center{\includegraphics[width=0.35\linewidth]{demo}}
	% \caption{Example image.}
	\label{fig:speciation}
\end{figure}


%  to make a guard page
\newpage

% generate the summary table
\tableofcontents
\newpage

%%%%%%%%%%%%%%%%%%%%%%%%%%%%%%%%%%%%%%%%%% INTRO %%%%%%%%%%%%%%%%%%%%%%%%%%%%%%%%%%%%%%%%%%%%
\section{Introduction}\label{Introduction}
% \footnote{\href{link}{text\_link}}
Le but de ce projet est de concevoir un contrôleur de refroidissement pour un système embarqué. Notre système est composé d'un NanoPi NEO Plus2\footnote{\href{https://wiki.friendlyelec.com/wiki/index.php/NanoPi_NEO_Plus2}{https://wiki.friendlyelec.com/wiki/index.php/NanoPi\_NEO\_Plus2}} et d'un module NanoHat OLED\footnote{\href{https://www.friendlyelec.com/index.php?route=product/product&product_id=191&search=NanoPi+NEO+Plus2&description=true&category_id=0&sub_category=true}{https://www.friendlyelec.com/index.php?route=product/product\&product\_id=191\&search=NanoPi+NEO+Plus2\&\\description=true\&category\_id=0\&sub\_category=true}} comportant un Display OLED. Notre système ne dispose pas d'un refroidissement actif, comme un ventilateur, nous simulons ce refroidissement par le clignotement d'une LED.\\

Notre projet est composé de 2 parties : \\
\begin{itemize}
	\item Un module kernel qui permet de contrôler la fréquence de la LED, de changer le mode de fonctionnement du module kernel, ainsi que de récupérer les informations sur la température du CPU.
	\item Un démon qui permet d'afficher les informations sur l'écran, et de contrôler le module kernel.
\end{itemize}


%%%%%%%%%%%%%%%%%%%%%%%%%%%%%%%%%%%%%%%%%% MODULE KERNEL %%%%%%%%%%%%%%%%%%%%%%%%%%%%%%%%%%%%
\section{Module Kernel}\label{ModuleKernel}
Ce module kernel a pour but de gérer le refroidissement de notre système embarqué, qui est simulé par le clignotement d'une LED. Le module propose deux modes de fonctionnement :
\begin{itemize}
	\item Manuel : L'utilisateur peut choisir la fréquence de clignotement de la LED.
	\item Automatique : La fréquence de clignotement de la LED est calculée en fonction de la température du microprocesseur.
\end{itemize}

Le module met à disposition une interface de contrôle à travers $sysfs$ permettant de modifier le mode de fonctionnement, la fréquence de la Modulation de largeur d'impulsion (PWM) et de lire la température du microprocesseur. \\

Toute la communication avec le module se fait à travers les attributs de la class. Nous en mettons trois à disposition :
\begin{itemize}
	\item $auto\_config$ : Permets de lire et modifier le mode de fonctionnement du module. Il peut prendre deux valeurs : $0$ pour le mode manuel et $1$ pour le mode automatique.
	\item $frequency\_Hz$ : Permets de lire et modifier la fréquence en Hz du PWM si le mode manuel est activé. La valeur $0$ permet d'arrêter le PWM et toute autre valeur positive fixe la fréquence du PWM, 
	\item $temperature\_mC$ : Permets de lire la température du microprocesseur en millidegré Celsius.
\end{itemize}

La mise en place des attributs se fait de la manière suivante :
\begin{lstlisting}[style=CStyle]
DEVICE_ATTR(frequency_Hz, 0664, show_frequency_Hz, store_frequency_Hz);
DEVICE_ATTR(auto_config, 0664, show_auto_config, store_auto_config);
DEVICE_ATTR(temperature_mC, 0444, show_temperature_mC, NULL);

static int __init my_module_init(void){
	/* ... */
	device_create_file(my_device, &dev_attr_frequency_Hz);
	device_create_file(my_device, &dev_attr_auto_config);
	device_create_file(my_device, &dev_attr_temperature_mC);
	/* ... */
}
\end{lstlisting}

Dans le code ci-dessus, les fonctions $show\_*()$ et $store\_*()$ sont des fonctions qui permettent de lire et modifier les attributs. Elles sont appelées automatiquement par le système d'exploitation lorsqu'un utilisateur lit ou modifie un attribut. Elles sont définies selon le modèle suivant :

\begin{lstlisting}[style=CStyle]
ssize_t store_attr(struct device *dev, struct device_attribute *attr, const char *buf, size_t count){
	int new_value;
	sscanf(buf, "%d", &new_value);
	my_device_attribute.value = new_value;
	return count;
}


ssize_t show_attr(struct device *dev, struct device_attribute *attr, char *buf){
	sprintf(buf, "%d\n", my_device_attribute.value);
	return strlen(buf);
}
\end{lstlisting}

Étant donné que ces attributs permettent de contrôler le module, une logique supplémentaire est intégrée dans les fonctions ci-dessus, afin de contrôler la cohérence des valeurs. Par exemple, la fréquence ne peut pas être modifiée si le module est en mode auto ou être négative. \\

Le module utilise deux timers indépendants pour gérer le clignotement de la LED et la lecture de la température. Les deux timers sont périodiques, le premier est configuré avec la fréquence donnée par l'attribut et le second avec une période de 500 millisecondes. \\
Pour les configurer, nous utilisons une structure $timer\_list$ qui contient les informations nécessaires à la gestion du timer. Nous utilisons la fonction $setup\_timer()$ pour initialiser la structure avec la callback voulue et la fonction $mod\_timer()$ pour démarrer le timer. Ce timer a comme unité de temps les $jiffies$ qui est une unité de temps définie par le système d'exploitation et représente le temps entre deux ticks d'horloge successives. Voici l'exemple de configuration du timer de température :
\begin{lstlisting}[style=CStyle]
static struct timer_list timer_temprature;

void run_timer(struct timer_list *timer, unsigned long period_us){
	/* ... */
	mod_timer(timer, jiffies + usecs_to_jiffies(period_us));
}

void timer_temprature_callback(struct timer_list *timer){
	/* ... */
	// run timer again
	run_timer(&timer_fan, my_device_attribute.period_us_d2);		
}

static int __init my_module_init(void){
	/* ... */
    timer_setup(&timer_temprature, timer_temprature_callback, 0);
    run_timer(&timer_temprature, TEMP_PERIOD_US);
	/* ... */
}
\end{lstlisting}

Les callbacks des timers nous permettent de mettre à jour la température avec laquelle nous recalculons la nouvelle fréquence du PWM et de créer un signal PWM avec un duty cycle de 50\% et la fréquence voulue. \\
Nos deux callbacks sont les suivantes :
\begin{lstlisting}[style=CStyle]
void timer_temprature_callback(struct timer_list *timer){
	struct thermal_zone_device *tzd;
	int temperature, ret, last_f;
	run_timer(&timer_temprature, TEMP_PERIOD_US);

	tzd = thermal_zone_get_zone_by_name("cpu-thermal");
	ret = thermal_zone_get_temp(tzd, &temperature);
	if(ret){
		pr_err("Pilotes Fan_ctl: Error in thermal_zone_get_temp\n");
		return;
	}
	my_device_attribute.temperature_mC = temperature;
	if(my_device_attribute.auto_config){
		last_f = my_device_attribute.frequency_Hz;
		if(temperature < TEMP_THR_1){
			my_device_attribute.frequency_Hz = TEMP_FRQ_1;
		}else if(temperature < TEMP_THR_2){
			my_device_attribute.frequency_Hz = TEMP_FRQ_2;
		}else if(temperature < TEMP_THR_3){
			my_device_attribute.frequency_Hz = TEMP_FRQ_3;
		}else{
			my_device_attribute.frequency_Hz = TEMP_FRQ_4;
		}
		my_device_attribute.period_us_d2 = S_IN_US / (2 * my_device_attribute.frequency_Hz);
		if(last_f == 0){
			run_timer(&timer_fan, my_device_attribute.period_us_d2);
		}
	}
}
void timer_fan_callback(struct timer_list *timer){
	static int state = 0;
	if(my_device_attribute.period_us_d2 == 0){
		state = 0;
	}else{
		run_timer(&timer_fan, my_device_attribute.period_us_d2);
		state = (state + 1) % 2;
	}
	gpio_set_value(GPIO_FAN, state);
}
\end{lstlisting}

La fonction $thermal\_zone\_get\_zone\_by\_name()$ permet de récupérer la zone thermique du CPU. La fonction $thermal\_zone\_get\_temp()$ permet de récupérer la température de la zone thermique, que nous stockons dans l'attribut $temperature\_mC$. \\

%%%%%%%%%%%%%%%%%%%%%%%%%%%%%%%%%%%%%%%%%% DAEMON %%%%%%%%%%%%%%%%%%%%%%%%%%%%%%%%%%%%%%%%%%%%
\section{Démon (Daemon)}\label{Démon (Daemon)}
Un démon (daemon) est un programme qui s'exécute en arrière plan. 
Son objectif est de gérer l'affichage des données sur l'écran, tel que la température, la fréquence, et le mode de fonctionnement du module kernel.
Pour faire cela, il dispose de deux interfaces de contrôle : \\
\begin{itemize}
	\item Physique : les 3 boutons présents sur la carte pour augmenter ou diminuer la fréquence et changer le mode de fonctionnement.
	\item Logiciel : un IPC (Inter Process Communication) pour recevoir des requêtes. Dans notre cas, nous choisissons d'utiliser un socket.
\end{itemize}

Nous réutilisons le code fourni dans le cours comme point de départ pour la création du démon. \\

Nous répartissons le code de notre programme dans différentes librairies afin de le rendre plus lisible. Voici les différents fichiers du démon et leur contenue : \\
\begin{itemize}
	\item \textit{main.c}\\
	Ce fichier contient le main de notre programme, il initialise le démon et contient le thread de réception des messages du socket.
	\item \textit{daemonfanlib.c/h}\\
	Cette librairie contient les fonctions nécessaires à l'initialisation du socket et la création du démon, ainsi que la constante qui indique le port du socket.
	\item \textit{gpio\_utility.c/h}\\
	Cette librairies permet d'accéder aux GPIOs de la carte, ainsi qu'aux informations fournis par le module kernel. Le header contient diverses constantes qui permettent de définir les GPIOs utilisés.
	\item \textit{ssd1306.c/h}\\
	Cette librairie fournis les fonctions utiles à l'utilisation de l'écran. Cette librairie nous a été fournie dans le cours.

\end{itemize}
\subsection{IPC Server}\label{IPCServer}
La communication entre un processus externe et le démon se fait par un socket, ce qui permet de contrôler la carte depuis n'importe quel ordinateur connecté au même réseau. 
L'initialisation du socket se fait dans la fonction \textit{void initSocket(int *mode, int *freq, pthread\_t *thread\_id, void* (*threadFunc)(void*))} qui est appelée dans la fonction \textit{main}.
La fonction prend un pointeur vers le mode, la fréquence, et l'id du thread, ainsi qu'un pointeur vers la fonction qui sera exécutée par le thread, afin de traiter les données reçues.
Le socket est initialisé avec l'adresse IP de la carte, et le port 8080. \\

\newpage
Comme nous sommes libres pour le choix du protocole de communication, nous choisissons d'utiliser un protocole simple pour contrôler le mode et la fréquence du module kernel.
Le protocole est composé de deux commandes : \\
\begin{itemize}
	\item M\textit{X} : Permets de choisir le mode de fonctionnement du module kernel. \textit{X} peut prendre 2 valeurs : 0 pour le mode manuel, et 1 pour le mode automatique.
	\item F\textit{XXX} : Permets de changer la fréquence de clignotement de la LED. \textit{XXX} est un nombre entre 0 et 999.
\end{itemize}

Par exemple, envoyer "M0", puis "F5" a pour effet de mettre le mode manuel et la fréquence de clignotement à 5 Hz.

Voici le code du thread qui s'occupe de traiter les données reçues par le socket : \\
\begin{lstlisting}[style=CStyle]
static void *threadSocket(void *arg)
{
	int client_fd = 0;
	char buffer[SOCKET_BUFFER_SIZE] = {0};
	// get the parameters
	socketParamThread *param = (socketParamThread*) arg;
	int addresslen = sizeof(param->address);
	//listen on the socket
	if((client_fd = accept(param->server_fd, (struct sockaddr*)&param->address, ((socklen_t*) &addresslen))) < 0) {
		syslog(LOG_ERR, "accept");
		exit(EXIT_FAILURE);
	}

	syslog(LOG_INFO, "threadSocket started\n");
	while(1) {
		int valread = read(client_fd, buffer, SOCKET_BUFFER_SIZE);
		if (valread == 0) {
			syslog(LOG_INFO, "client disconnected\n");
			close(client_fd);
			client_fd = accept(param->server_fd, (struct sockaddr*)&param->address, ((socklen_t*) &addresslen));
		} else {
			if (buffer[0] == 'M') {
				*param->mode = buffer[1] - '0';
				writeMode(*param->mode);
				if (*param->mode == 0) {
					writeFreq(*param->freq);
				}
			} else if (buffer[0] == 'F') {
				*param->freq = atoi(&buffer[1]);
				writeFreq(*param->freq);
			}
			syslog(LOG_INFO, "received: %s\n", buffer);
		}
	}
	free(param);
	return NULL;
}
\end{lstlisting}
Le thread attend qu'un client se connecte, puis lit les données reçues et les traite.
Si le client se déconnecte, le thread attend qu'un nouveau client se connecte.
Le code ne permet pas à plusieurs clients de se connecter simultanément, mais il est possible de le faire en créant un thread qui attend des connexions et crée un nouveau thread pour chaque client qui se connecte. \\

\newpage
\subsection{IPC Client}\label{IPCClient}
Le contrôle du démon se fait par un petit programme C qui permet d'envoyer des commandes au démon.
Nous avons écrit une petite librairie \textit{command.c} qui contient les commandes suivantes pour contrôler le démon en passant par le socket. Voici le header de la librairie : \\
\begin{lstlisting}[style=CStyle]
#define DAEMON_ADDR "127.0.0.1"
#define DAEMON_PORT 8080

void init_socket();
void send_mode(int mode);
void send_freq(int freq);
void close_socket();
\end{lstlisting}

Pour utiliser cette librairie, il suffit d'appeler la fonction \textit{init\_socket()} au début du programme.
L'appel des fonctions \textit{send\_mode()} et \textit{send\_freq()} permet d'envoyer des commandes au démon.
La fonction \textit{close\_socket()} permet de fermer le socket. \\
Voici un exemple d'utilisation de la librairie : \\
\begin{lstlisting}[style=CStyle]
int main()
{
	// try to pass to manual mode
	init_socket();
	// set mode to manual
	send_mode(0);
	// set freq to 20Hz
	send_freq(20);
	usleep(1000000); // wait 1s
	// set freq to 2Hz
	send_freq(2);
	// wait 2s
	usleep(2000000);
	// set mode to auto
	send_mode(1);
	close_socket();
	return 0;
}
\end{lstlisting}

Ce programme permet de passer le démon en mode manuel et changer la fréquence de clignotement de la LED, puis de passer en mode automatique.

\subsection{Contrôle de la LED rouge}\label{Leds}
L'accès à la LED par le démon se fait par l'interface sysfs.
Pour cela, nous avons créé une fonction qui permet d'initialiser la LED et une fonction qui permet de choisir l'état de la LED rouge.
Le prototype de la fonction d'écriture est : \textit{void writeLed(int value)}.
Voici le code d'initialisation de la LED : \\
\begin{lstlisting}[style=CStyle]
void initLeds()
{
	int f = open(GPIO_UNEXPORT, O_WRONLY);
	write(f, LED, strlen(LED));
	close(f);

	// export pin to sysfs
	f = open(GPIO_EXPORT, O_WRONLY);
	write(f, LED, strlen(LED));
	close(f);

	// config pin
	f = open(GPIO_LED "/direction", O_WRONLY);
	write(f, "out", 3);
	close(f);
	g_led_fd = open(GPIO_LED "/value", O_WRONLY);
	syslog(LOG_INFO, "leds initialized\n");
}
\end{lstlisting}
Le code commence par exporter la LED, puis configure la LED en sortie, en écrivant "out" dans le fichier direction.
Enfin, il ouvre le fichier value, qui permet d'écrire dans la valeur de la LED. Le descripteur de fichier est stocké dans la variable globale \textit{g\_led\_fd}. \\
Pour finir un simple appel à la fonction \textit{void writeLed(int value)} permet de changer l'état de la LED rouge.


\subsection{Bouton}\label{Bouton}
Pour réaliser cette partie du TP, nous réutilisons en partie le code que nous avons écrit pour le rendu précédent.
Notre code utilise les epoll pour gérer les interruptions des boutons.
Afin de rendre le code main plus lisible, nous avons écrit une fonction \textit{int initButtonsAndTimer()} qui permet d'initialiser les boutons, et un timer pour être utilisé avec les epoll.
Cette fonction fait appel aux fonctions epoll\_create1(), epoll\_ctl(), et timerfd\_create() pour créer les epoll et le timer.
Elle retourne un descripteur de fichier qui permet de lire les évènements sur les boutons, et le timer.
\begin{lstlisting}[style=CStyle]
// init the buttons S1, S2 and S3, and the timer
epfd = initButtonsAndTimer();
\end{lstlisting}
L'attente des évènements sur les boutons s'effectue dans le main avec la fonction \textit{epoll\_wait()}.
Voici un extrait du code main, qui permet de lire les évènements sur les boutons : \\
\begin{lstlisting}[style=CStyle]
while (1) {
	struct epoll_event event_arrived[NUM_EVENTS];
	syslog(LOG_INFO, "waiting for event epoll\n");
	writeLed(LED_OFF);
	int nr = epoll_wait(epfd, event_arrived, NUM_EVENTS, -1);
	syslog(LOG_INFO, "event arrived\n");
	if (nr == -1) {
		// printf("error epoll_wait: %s\n", strerror(errno));
		syslog(LOG_ERR, "epoll_wait");
		exit(EXIT_FAILURE);
	}
	for(int i = 0; i < nr; i++){
		my_context *ctx = event_arrived[i].data.ptr;

		switch (ctx->ev){
		case EV_BTN_1: // increase frequence
			syslog(LOG_INFO, "button 1 pressed\n");
			if(ctx->first_done == 0){
				ctx->first_done = 1;
				break;
			}
			freq++;
			writeFreq(freq); // will fail if in auto mode
			writeLed(LED_ON);
		break;
		// ...
\end{lstlisting}

Dans l'epoll, nous ajoutons un timer, qui permet de rafraichir l'affichage de la température et de la fréquence en mode automatique.
Le timer est initialisé avec une période de 200ms, il est possible de le modifier dans le fichier gpio\_utility.h, en modifiant la constante \textit{DEFAULT\_PERIOD}.
Voici un extrait de la gestion de l'évènement timer : \\
\begin{lstlisting}[style=CStyle]
case EV_TIMER:
syslog(LOG_INFO, "timer expired\n");
updateTempCPU();
if(mode == 1) // if in auto mode
{
	// read the actual freq
	freq = readFreq();
	// show it on the screen
	writeLCDFreq(freq); 
}
else // if in manual mode
{
	// syslog(LOG_INFO, "manual mode\n");
}
break;
\end{lstlisting}

\newpage
\subsection{Concurence}
Comme notre programme est multithreadé (thread socket et thread main), il est possible d'avoir des problèmes de concurrence. Pour éviter cela, nous ajoutons un mutex sur les commandes envoyées au LCD.
Les variables de mode et de fréquence sont aussi partagées. Elles peuvent être modifiées par le thread socket et par le main. Mais les risques d'accès simultané sont assez faibles, il faudrait qu'une requête soit envoyée au moment ou quelqu'un écrit sur le socket. L'accès à ces variables devrait également être protégé par un mutex.
Exemple de protection dans la fonction \textit{void writeLCDFreq(int freq)} : \\
\begin{lstlisting}[style=CStyle]
void writeLCDFreq(int freq)
{
    char str[32] = {0};
    sprintf(str, "Freq: %03d Hz", freq);
    pthread_mutex_lock(&g_mutex_lcd);
    ssd1306_set_position (0 ,4);
    ssd1306_puts(str);
    pthread_mutex_unlock(&g_mutex_lcd);
}
\end{lstlisting}
Sans cette protection, deux threads essayaient d'écrire en même temps sur le LCD, et le texte n'était pas affiché correctement.

\section{Conclusion}

Notre programme fonctionne correctement, le rafraichissement de la température est bien effectué et la fréquence est actualisé en mode automatique.
Le contrôle de la fréquence en mode manuel fonctionne correctement. Nous affichons également une ligne supplémentaire pour indiquer le mode actuel.
Il est possible de contrôler notre démon avec un socket client sur le réseau pour changer le mode et la fréquence. Ce qui peut être une fonctionnalité intéressante ou un problème de sécurité. Nous pouvons mettre une règle dans le pare-feu, pour n'autoriser que certaines adresses IP à se connecter au socket.
\linebreak
\linebreak
Pour garder un code propre, nous voulions éviter l'utilisation d'un timer dans le démon pour le rafraichissement de l'affichage. Un timer étant déjà présent dans le module kernel, cette solution nous semblait redondante. Nous avons donc cherché à utiliser les epoll pour détecter les changements de valeur des variables de température et de fréquence sur les attributs de /sys/class comme cela est fait pour les GPIOs. Cependant, nous n'avons pas trouvé de solution pour une telle implémentation. Nous avons donc opté pour un timer, qui permet de rafraichir l'affichage toutes les 200ms.
Nous avons trouvé ce projet intéressant, il nous a permis de mettre en pratique beaucoup de concept vu en cours et de nous familiariser avec les démons.

%----------------------------------------------------------------------------------------
%	BIBLIOGRAPHY
%----------------------------------------------------------------------------------------

\printbibliography % Output the bibliography

%----------------------------------------------------------------------------------------

\end{document}